%% For double-blind review submission
\documentclass[sigplan,10pt,review]{acmart}\settopmatter{printfolios=true}
%% For single-blind review submission
%\documentclass[acmsmall,10pt,review]{acmart}\settopmatter{printfolios=true}
%% For final camera-ready submission
%\documentclass[acmsmall,10pt]{acmart}\settopmatter{}

%% Note: Authors migrating a paper from PACMPL format to traditional
%% SIGPLAN proceedings format should change 'acmsmall' to
%% 'sigplan'.


%% Some recommended packages.
\usepackage{booktabs}   %% For formal tables:
                        %% http://ctan.org/pkg/booktabs
\usepackage{subcaption} %% For complex figures with subfigures/subcaptions
                        %% http://ctan.org/pkg/subcaption


\makeatletter\if@ACM@journal\makeatother
%% Journal information (used by PACMPL format)
%% Supplied to authors by publisher for camera-ready submission
\acmJournal{PACMPL}
\acmVolume{1}
\acmNumber{1}
\acmArticle{1}
\acmYear{2017}
\acmMonth{1}
\acmDOI{10.1145/nnnnnnn.nnnnnnn}
\startPage{1}
\else\makeatother
%% Conference information (used by SIGPLAN proceedings format)
%% Supplied to authors by publisher for camera-ready submission
\acmConference[CHESE'17]{CHESE'17}{October 24, 2017}{Vancouver, Canada}
\acmYear{2017}
\acmISBN{978-x-xxxx-xxxx-x/YY/MM}
\acmDOI{10.1145/nnnnnnn.nnnnnnn}
\startPage{1}
\fi


%% Copyright information
%% Supplied to authors (based on authors' rights management selection;
%% see authors.acm.org) by publisher for camera-ready submission
\setcopyright{none}             %% For review submission
%\setcopyright{acmcopyright}
%\setcopyright{acmlicensed}
%\setcopyright{rightsretained}
%\copyrightyear{2017}           %% If different from \acmYear


%% Bibliography style
\bibliographystyle{ACM-Reference-Format}
%% Citation style
%% Note: author/year citations are required for papers published as an
%% issue of PACMPL.
\citestyle{acmauthoryear}   %% For author/year citations


\usepackage[T1]{fontenc}
\usepackage[utf8]{inputenc}

\usepackage{pifont}
\def\Deliverables{\noindent\ding{229} \textbf{Deliverables}}
\def\MembersExperience{\noindent\ding{251} \textbf{Members' experience}}
\def\Challenge#1{\noindent\ding{237} \textit{#1}}
\def\Check{\ding{51}}
\def\NoCheck{\ding{55}}

\usepackage{tabularx}
\newcolumntype{C}{>{\centering\arraybackslash}X}
\newcolumntype{L}{>{\raggedright\arraybackslash}X}
\newcolumntype{R}{>{\raggedleft\arraybackslash}X}
\newcommand*\rot{\rotatebox[origin=B]{90}}
\usepackage{booktabs}

\usepackage[nohyperlinks,nolist]{acronym} % options: nolist printonlyused



\begin{document}

%% Title information
\title[Peer-Testing]{Development of a Web Platform for Code Peer-Testing}         %% [Short Title] is optional;
                                        %% when present, will be used in
                                        %% header instead of Full Title.
%\titlenote{with title note}             %% \titlenote is optional;
                                        %% can be repeated if necessary;
                                        %% contents suppressed with 'anonymous'
%\subtitle{Subtitle}                     %% \subtitle is optional
%\subtitlenote{with subtitle note}       %% \subtitlenote is optional;
                                        %% can be repeated if necessary;
                                        %% contents suppressed with 'anonymous'


%% Author information
%% Contents and number of authors suppressed with 'anonymous'.
%% Each author should be introduced by \author, followed by
%% \authornote (optional), \orcid (optional), \affiliation, and
%% \email.
%% An author may have multiple affiliations and/or emails; repeat the
%% appropriate command.
%% Many elements are not rendered, but should be provided for metadata
%% extraction tools.

%% Author with single affiliation.
\author{Manuel Maarek}
%\authornote{with author1 note}          %% \authornote is optional;
                                        %% can be repeated if necessary
\orcid{0000-0001-6233-6341} % Maarek             %% \orcid is optional
\affiliation{
%  \position{Assistant Professor}
  \department{School of MACS}              %% \department is recommended
  \institution{Heriot-Watt University}            %% \institution is required
%  \streetaddress{Riccarton}
  \city{Edinburgh}
  \state{Scotland}
%  \postcode{EH14 4AS}
  \country{UK}
}
\email{M.Maarek@hw.ac.uk}          %% \email is recommended

%% Author with two affiliations and emails.
\author{Léon McGregor}
%\authornote{with author2 note}          %% \authornote is optional;
                                        %% can be repeated if necessary
%\orcid{nnnn-nnnn-nnnn-nnnn}             %% \orcid is optional
\affiliation{
%  \position{Student}
  \department{School of MACS}              %% \department is recommended
  \institution{Heriot-Watt University}            %% \institution is required
%  \streetaddress{Riccarton}
  \city{Edinburgh}
  \state{Scotland}
%  \postcode{EH14 4AS}
  \country{UK}
}
%\email{first2.last2@inst2a.com}         %% \email is recommended


%% Paper note
%% The \thanks command may be used to create a "paper note" ---
%% similar to a title note or an author note, but not explicitly
%% associated with a particular element.  It will appear immediately
%% above the permission/copyright statement.
%\thanks{QAA Heriot-Watt}                %% \thanks is optional
                                        %% can be repeated if necesary
                                        %% contents suppressed with 'anonymous'


%% Abstract
%% Note: \begin{abstract}...\end{abstract} environment must come
%% before \maketitle command
\begin{abstract}
  As part of formative and summative assessments in programming
  courses, students work on developing programming artifacts following
  a given specification. These artifacts are evaluated by the
  teachers. As the end of the evaluation, the students receive
  feedback and marks. Providing feedback on programming artifacts is
  time demanding and could make the feedback to arrive late for it to
  be effective for the student's learning. Peer-feedback has been
  praised for offering a timely and effective learning activity.

  In this paper we report on the development of a Web platform for
  peer-feedback on programming artifacts through program testing. We
  discuss the development process of our peer-testing platform
  informed by teachers' and students perspectives.
\end{abstract}

%% 2012 ACM Computing Classification System (CSS) concepts
%% Generate at 'http://dl.acm.org/ccs/ccs.cfm'.
\begin{CCSXML}
<ccs2012>
<concept>
<concept_id>10003456.10003457.10003527.10003531.10003751</concept_id>
<concept_desc>Social and professional topics~Software engineering education</concept_desc>
<concept_significance>500</concept_significance>
</concept>
<concept>
<concept_id>10003456.10003457.10003527.10003531.10003533</concept_id>
<concept_desc>Social and professional topics~Computer science education</concept_desc>
<concept_significance>300</concept_significance>
</concept>
<concept>
<concept_id>10011007.10011074.10011099.10011102.10011103</concept_id>
<concept_desc>Software and its engineering~Software testing and debugging</concept_desc>
<concept_significance>500</concept_significance>
</concept>
<concept>
<concept_id>10010405.10010489.10010492</concept_id>
<concept_desc>Applied computing~Collaborative learning</concept_desc>
<concept_significance>300</concept_significance>
</concept>
</ccs2012>
\end{CCSXML}

\ccsdesc[500]{Social and professional topics~Software engineering education}
\ccsdesc[300]{Social and professional topics~Computer science education}
\ccsdesc[500]{Software and its engineering~Software testing and debugging}
\ccsdesc[300]{Applied computing~Collaborative learning}
%% End of generated code


%% Keywords
%% comma separated list
\keywords{peer-testing, peer-feedback, software testing}  %% \keywords is optional


%% \maketitle
%% Note: \maketitle command must come after title commands, author
%% commands, abstract environment, Computing Classification System
%% environment and commands, and keywords command.
\maketitle


\section{Introduction}





\cite{GroHamKumMaaMcGrShaWelZan_STEM-HE-2017}
\cite{McGregor_BSc-2017}


\subsection{Peer-Feedback and Peer-Assessment}


\paragraph{What is it?}

Peer assessment is a process in which students assess each other. This
opposes the more traditional stance where a teacher performs the
assessment. As defined by \citet{topping_peer_2009},
\begin{quote}
  ``Peer assessment is an arrangement for learners to consider and
  specify the level, value, or quality of a product or performance of
  other equal-status learners.''
\end{quote}
That is to say, students with a similar level of education assessing
the work of each other to give critical feedback and discussion. This
could be done in many ways, such as between pairs or in groups, and
can be performed on any number of different activities from
programming exercises to oral reports.

\paragraph{Why use it?}
Peer assessment is a process with many benefits to participants in
education. \citet{sadler_impact_2006} have suggested the
following concepts that peer assessment can help with
\begin{itemize}
\item Peer assessment is more immediate, so students can get more
  feedback, and sooner;
\item Students performing marking can reduce the workload for
  teachers;
\item The process of checking and thinking about another students
  answer can improve a students own understanding;
\item Peer assessment can help students better understand testing and
  can become aware of their own strengths and weaknesses;
\item Following peer assessment students can gain an improved attitude
  towards the process of learning as a whole.
\end{itemize}

Peer assessment can offer much help towards education of students, but
it would be worthwhile to know just which aspects are the most
useful. A study conducted by \citet{li_assessor_2010}
investigated the peer assessment process with the aim of discovering
which part of it is most useful to the students involved: being an
assessor or being assessed. To study this, undergraduate student teachers
were given the task of creating a \textit{WebQuest project}\footnote{In this
  context, a \textit{WebQuest} project is an activity created by the
  student teachers that would be given to their students. Following
  the \textit{WebQuest} instructions, students are guided through
  Internet resources, and offered ``scaffolding activities'' to help
  them learn.}. This was then marked by independent assessors,
and the student teachers were given a chance to provide feedback on
other student teachers' \textit{WebQuest}s. Following this, the
feedback was returned and students teachers were given another chance
to improve their project, and it was marked again. The quality of the
peer feedback itself was also checked by the independent markers. The
study found that

\begin{quote}
  ``there was a significant relationship between the quality of the
  peer feed-back the students provided for others and the quality of
  the students' own final projects''
\end{quote}

The findings of the investigation would suggest that the actual
exercise of providing feedback to others (acting as an assessor) is a
worthwhile process for learning from. This study also concluded that
there was no reasonable link between the feedback itself as a learning
tool, suggesting that the act of giving feedback itself is more
valuable and that low quality feedback does not harm the learning
experience.

\paragraph{How to do it?}

With the knowledge that peer assessment can be useful, it is important
to know how a peer assessment should be conducted. A study performed
in a classroom environment by \citet{smith_using_2012}
focused more on the use of peer assessment as a tool for teaching
testing of code, in addition to the existing course.

Over the course of this study, which took place using coursework from
a 12-week university course, the following was completed for each
coursework: submitting solutions, then submitting peer reviews (which
includes a description of the testing that they performed on another
solution, and the results of this), and then a review of the peer
assessment (including what was learnt, an evaluation of feedback on
their own solution, and optionally a corrected solution).

One particularly noteworthy aspect of this use of peer assessment was
the double-blind nature, ensuring anonymity. Students would not be
aware of who they were marking, or were marked by. To enforce this
completely, submitted code was obfuscated (Java sources into Byte
code). One advantage of this is that it strips out identifying
variable names and comments, which could identify other
students. However, a downside of using byte code is that it can make
it difficult to do in-depth analysis of the source structure which may
make it harder to write complete test cases.

The study identified two key features that assignments for peer
assessment need to have: a well-defined interfaces, and freedom for
implementation.  In addition to this, \citet{smith_using_2012} has
found that it was possible to integrate the peer assessment process
without having to significantly alter the existing course material,
and the students taking part enjoyed the experience. This shows
promise, as it could indicate many \ac{CS} courses (that
offer coursework meeting the requirements), could be modified to
include their own peer assessment exercises.

Peer assessment can prove to be a very valuable experience for
students. \citet{falchikov_improving_2013} has collected various case
studies of past peer assessments, and the following aspects can be
found:
\begin{itemize}
\item If the marking criteria are properly explained, there is often
  no significant difference between the marks awarded by students and
  those that would be awarded by teachers. This would tend to indicate
  that students do assess each other fairly.
\item One of the most important aspects of peer assessment is the
  ability of the student to learn how to assess other students and
  from this learn how to critically assess and improve their own work.
\item It is important to make sure students feel confident, otherwise
  they may not assess their peers as honestly as they might otherwise
  have done. Some students will feel conflicted about marking their
  peers, particularly if they might have to give low marks.
\item During peer assessment more benefits may come from students
  assessing multiple solutions, rather than each focusing on one.
\end{itemize}

% \paragraph{Summary of Peer Assessment}
% The process of peer assessment holds much value for all of the
% participants in the educational process, and can provide some skills
% that can be carried beyond students time in education. It can be used
% to provide more immediate feedback, help students to learn as they
% evaluate solutions and create this feedback, and could be integrated
% with some existing coursework assignments. Beyond just letting
% students learn about how to grade against a mark scheme, peer
% assessment can often teach students a lot about the subject material
% itself, letting them gain a deeper understanding about it. Peer
% assessment also helps to grow students ability to critically evaluate
% both their own, and others, work. Students who are to engage in Peer
% assessment should be given sufficient help so that they can perform
% the assessment with confidence, and work with multiple peers to get
% the full positive effects of peer assessment.


\section{Informed Peer-Testing Requirements}

\subsection{Initial Requirements}

\begin{itemize}
\item Each student can submit a solution
\item Students are paired before the system enters in peer-testing mode
\item When entering the peer-testing mode, each student can submit
  tests, which are run against her peer's solution.
\item Student can submit a test with feedback to her peer.
\end{itemize}

\subsection{Focus Group Discussions with Teachers}



\begin{itemize}
\item Teachers' feedback and marking of tests and feedback submitted
  by students. The tester gives feedback, and then the developer responds how they would take this on board. Then the teacher could give their input as well (either individually or on a per-test basis).
Feedback could be given after testing has been done within groups of students 
  % item 1 in notes 2017-01-31
\item
  Ability to track the students' attempts at testing their own submission and their peer's submissions. A form of testing-based learners log. 
Tracking everything a student has done and putting this in one place could offer a way for a teacher to give feedback [on a given coursework] 
  % item 2 in notes 2017-01-31

\item

  Evaluation of tests submitted: evaluate the coverage of the tests submitted, evaluation of the performance in terms of memory and time, categorising test cases based on the specification (the categories could be provided). Suggestion to use a naming convension for tests. 
This was the suggestion that a pipeline system could be introduced
where uploaded files (either solutions or test cases) are analysed and
subsequently processed by plugins (existing framework for code
similarity could be used here). 
Another pipeline would be introduced for the actual running of tests where more data is collected. 

  % item 3 and 7 in notes 2017-01-31


\item
  Matching peers could be done automatically and then modified manually. Information provided or used for automation could be initial tests submitted, metrics on the submitted code, campus, gender. Peers could be organised in pairs or in groups. 

  This is out of the current platform

  % item 5 in notes 2017-01-31

\item Anonymity.



  Should peer-testing be anonymous? This would require to hide identifications in submitted files (or to ask the students to do so). This would allow for teacher-crafted submissions and tests to be included as peer submissions. 
An alternative is to sanitized the solutions submitted to keep identities anonymous.
Anonymity has a potential advantage which is that it includes gender
anonymity, and campus anonymity.

  % item 6 in notes 2017-01-31

\item
Testing owns implementation should be allowed form day 1. The ability to craft tests based on someone else's implementation could also be allowed prior to peer-testing (black-box testing). Some own tests could be made private so that they are not shared with peers. Tests could be crafted based on input only, the expected output is then produced by a standard solution.

Alternative option is to perform some default tests on uploaded
submissions. This would go beyond peer-testing.


  % item 8 in notes 2017-01-31

\item
  Tests as I/O vs Unit testing

  % item 8 in notes 2017-01-31

  
\item

  The coursework specification needs to provide interfaces where
  testing will be conducted. The coursework could be broken down into
  smaller bits and some small bits could be tested among the students
  (without individual teacher feedback).

    % item 9 in notes 2017-01-31

\item
  Testing framework should not hide the technicality from the
  students: commandline ran should be displayed, raw submission should
  be allowed even if they fail preset verifications (message should indicate the issues and some change suggestions).


      % item 10 in notes 2017-01-31

\end{itemize}


\subsubsection{Software Testing}

%% Mostly from McGregor_BSc-2017

When considering the testing of software, there are several ways that
this can be done. To produce a We platform that enables peer-assessment,
an appropriate testing methodology will need to be selected. Based on
suggested testing styles from  \citet{laboon_friendly_2016},
some testing methodologies were considered for inclusion:
\begin{description}
\item[Linting] Performing basic checks for 'smells' of bad code
  design, such as identifying unused variables or methods. Not
  particularly useful from a perspective of checking code is correct,
  but is still useful in terms of producing good quality code.
\item[Unit testing] The use of xUnit style tests. Often used in
  \ac{TDD}, these could be useful in detecting flaws if
  written post-development by a peer assessor. This would need the
  assessor to be familiar with how to write xUnit style tests, and how
  they should be structured, or it would not be viable.
\item[Expected output testing] Running a program with some input and
  comparing the output to what is expected. The issue here is that the
  peer assessor first needs to know what the correct output is. Unlike
  unit testing, this would simply involve a series of inputs, and a
  series of expected outputs, and the task of checking the correctness
  from these tests is left to the website.
\item[Scenario Testing] Simulate an actual usage scenario. Assessors
  would use solutions as if they were third party libraries (a
  \emph{black box}), and develop their own programs that would use
  these as if in an actual use case. These programs would perform
  checks to make sure the solutions being tested were running as
  expected. This is more involved than unit or expected output
  testing, as it might be better placed to discover side effects of
  continued use of the solution.
\item[Property based testing] Using a proof checker or random testing
  such as QuickCheck to ensure that a program is acting correctly. This
  would place additional overhead onto the students as assessors, as
  they would need to learn and understand the annotations used by a
  proof checker, but could provide a lot of coverage of possible
  inputs.
\end{description}

\subsubsection{Anonymity}

%% Mostly from McGregor_BSc-2017

One issue that can arise when students are interacting with each other
from a perspective of evaluating each other is their anonymity. During
a peer assessment / automated testing experiment performed by
\citet{sitthiworachart_effective_2004}, the students remarked about
the use of anonymity. One students response suggested that in a
non-anonymous peer assessment exercise, their marking would be
influenced if they knew who it was they were marking. Another student
said they would decline to take part if the peer assessment was not
conducted anonymously. What can be drawn from these comments is that
anonymity is clearly very important to those taking part in the peer
assessment. But the opinions of anonymity might not correlate with the
actual effect it has.

To study the effect that anonymity has, as opposed to just opinions
about it, a study was conducted by \citet{li_role_2016}. This study
aimed to investigate just how effective anonymity is when it comes to
making peer assessment more effective, and whether any negative impact
from a lack of anonymity can be mitigated.
%
This quasi-experimental study was conducted with some in-training
teachers, and aimed to see which is the most effective method of /
using / while conducting a peer assessment exercise: Having assessors
and assessees know each others identities, remain anonymous, or know
identities while having received training. The training in this case
involved watching a video that described some of the stresses and
concerns that arise during peer assessment, and various forms of
discussion regarding this.
%
The study did not cover the case of being anonymous and getting
training. This was because the training was intended as a fallback for
when anonymity is not possible. But this does invite the question of
just how effective would anonymity be if training were offered as
well.
%
The study found: that ``anonymity improves student performance in peer
assessment'', that if anonymity cannot be guaranteed, negative effects
arising from this can be offset using training,and that anonymity does
not reduce the pressure and tension related with peer assessment.



\subsection{Experiment, Questionnaire and Focus Group Discussion with Students}



\begin{itemize}
\item Peer-feedback needs to be a peer-dialogue
\item Training needs to be provided to make sure the peer-testing
  platform is used effectively
\item ...
\end{itemize}


\section{Peer-Testing Platform}


\subsection{Interface between programming and testing artifacts}

\subsection{Peer-testing stages}


Table~\ref{tab:stages}

\begin{table*}
  \centering
  \begin{tabularx}{\hsize}{clc@{}c@{}c@{}cLL}
    \toprule
    \multicolumn{2}{l}{\bf Stage}
    & \rot{Solutions upload}
    & \rot{Tests upload}
    & \rot{Self-testing}
    & \rot{Peer-testing}
    & \bf Students
    & \bf Teachers \\
    % Stage & \rot{Vertical} &  \\
    \midrule
    0 & Coursework Setup
    & \NoCheck
    & \NoCheck
    & \NoCheck
    & \NoCheck
    & Not involved at this stage.
    & Setting Coursework specification, and program
    interfaces. Enrolling students and setting up
    peer-groups. Preparing oracle solution and tests. \\
    \cmidrule(rl){2-8}
    1 & Development \& Self-Testing
    & \Check
    & \Check
    & \Check
    & \NoCheck
    & Develop and upload solutions and tests. Run own tests on own
    solutions and on oracle solution. Run provided tests on own
    solutions.
    & Observing individual progress. Amending peer-groups. Adding
    extra tests or solutions. \\
    \cmidrule(rl){2-8}
    2 & Peer-Testing \& Feedback
    & \NoCheck
    & \Check
    & \Check
    & \Check
    & Run tests on peers' solutions, review peers' code to send
      feedback. Receive tests and feedback. Enter discussions with peers.
    & Monitoring students discussions. Prepare individual feedback. \\
    \cmidrule(rl){2-8}
    3 & Teachers' Feedback
    & \NoCheck
    & \NoCheck
    & \NoCheck
    & \NoCheck
    & Submit an experience report detailing how they would take
    on-board feedback they received to improve their program and how
    they compare their solution with their peers.
    & Provide group feedback at group level.
    \\
    \bottomrule
  \end{tabularx}
  \caption{Peer-testing stages implemented in the Web platform}
  \begin{minipage}{.8\linewidth}
    The table indicates for each stage whether students are allowed to
     submit
    artifacts solutions, to submit test cases, to test their own
    implementation, to view and test their peers solutions. It
    also describes the expected actions by students and teachers
    participants. 
  \end{minipage}
  \label{tab:stages}
\end{table*}

\subsection{Feedback discussion}


\subsection{Internal naming convention}

\section{Related Works}

\subsection{Peer-Feedback and Peer-Assessment}






%% Mostly from McGregor_BSc-2017

A study by \citet{davies_computerized_2000} was conducted using a peer
testing system in a university's communications and networking
module. This peer testing system aimed to help students easily provide
feedback on reports, in addition to attempting to locate plagiarism.

The \ac{CAP} provided an interface to let students read the report of
another student. They could then provide feedback on the report and
follow references on the report to identify any \textit{copy \& paste}
plagiarism.

\citet{davies_computerized_2000} conducted an experiment which
followed the results students achieved over a sequence of 4 tasks
(prepare a report, answer multiple choice test, engage in peer marking
of the reports, answer a final multiple choice test).  The experiment
looked to see if the peer marking was at all helpful in improving the
marks of students between the two tests. The results found that the
peer marking was very useful for students who had initially performed
poorly. Student comments on the experience would suggest that they
found the peer assessment process both enjoyable and
informative. There is also evidence from this feedback indicating the
importance of maintaining anonymity, as some students felt it would be
difficult to do the marking non-anonymously.  This study also revealed
that students learnt a lot from the ``repetitive nature of the
marking''.

Other Peer-Assessment systems
\cite{Aropa} \cite{WebPA}

%% Mostly from McGregor_BSc-2017

\citet{goldwasser_gimmick_2002} started a peer testing initiative on a
Data Structures course.  The main objective of this was to get
students taking the course to submit, along with coursework, a test
case to run on the program with the aim of finding bugs in other
students programs. The main objective in this case was to focus
learning on software testing, and the peer testing aspect was
secondary, however the study does hold many parallels with our
platform.

The submitted coursework solutions would be collected, along with each
test case. Then, using an automated system each of the test cases
would be run against each of the solutions. The major downside to this
being that there is $x^2$ complexity in terms of the submission count.

% \citet{goldwasser_gimmick_2002} offers some advice for potential
% future implementations of such an automated testing system:
% \begin{itemize}
% \item To prevent some students falling behind due to having buggy
%   input parsers, have the lecturer provide a standard implementation
%   to to all students. Similarly, output needs to be handled in a
%   standardised way.
%  \item A limit should be placed on how large a test case a student can
%    submit.
%  \item The automated testing system should be able to handle the case
%    where a solution works correctly, produces the wrong output,
%    crashes, or does not terminate.
%  \item Additionally, the testing system should be able to recover and
%    restart from a given position without having to re-run all test
%    cases, or re-run specific test cases if one of the submissions
%    should change
% \end{itemize}




%% Mostly from McGregor_BSc-2017

To determine whether students can benefit from peer assessment in an
online, \textit{technology-enhanced} environment, a research project
was conducted by \citet{keppell_peer_2006}. This project
involved some university lecturers, who re-structured their courses to
involve more peer learning, with the aid of a \ac{VLE}.

The project took place over three different case studies. The
different studies used journal, discussion and file sharing features
of the \ac{VLE} to enhance two courses in fashion design, art education
and the design of a new learning website. Based on the evidence
collected during these case studies, it was found that:
\begin{itemize}
\item Students found this assessment to be fairer than just against
  teacher;
\item Students appreciated the ongoing peer critique performed through
  reflective journals;
\item Teachers felt the instant nature of feedback was very useful;
\item The support offered by the assisting technology of the \ac{VLE}
  encouraged collaboration.
\end{itemize}
The authors of the project also suggest that unless the peer
assessment within the \acp{VLE} is a marked process (against a
students grade), students may be unwilling to participate.



%% Mostly from McGregor_BSc-2017

In the 1990s, the University of Nottingham produced \emph{Ceilidh}, a
system that had the intent of automating and improving the assessment
process for C Language Coursework. Later improvements were made by
Heriot-Watt University adding support for \ac{SML}
\cite{foubister_automatic_1997}, the system was later renamed
CourseMarker \cite{higgins_coursemarker_2003}. This tool was used to
guide students through the coursework, and then check the correctness
of submissions. Ceilidh offered a skeleton answer to students, which
pointed out any special language features that should be used. The
submitted solutions are checked for correctness with a mixture of
verifying output against a model solution, and checking the style of
the code (e.g line count, use of certain function names, etc).  In
particular for \ac{SML}, students were asked to provide the types for
written functions, as an additional test of understanding.

In their study, \citet{foubister_automatic_1997}
investigated if such a system used for automatic assessment of
students would be useful. To check this, the results of the final exam
mark were compared to those marks from previous years where Ceilidh
was not used. The study found that there was no real improvement made
by Ceilidh, but most importantly there were no detrimental
effects. Some positive side effects were observed: the time taken for
students to receive feedback on the coursework was markedly decreased,
and the tracking of progress offered by the tool allowed teachers to
more easily identify any students that were having serious
difficulties.

\subsection{Collaborative Programming Learning}

%% Mostly from McGregor_BSc-2017
As an aside to peer assessment, one optional feature of such an
implementation could be to offer automated testing. If a lecturer so
wished, they could use the system with a pre-submitted test case. In
effect, acting as a peer to each student, so that they could receive
test results immediately (or as quickly as the test completes) upon
submission.
%
The effectiveness of an automated testing tool (as a sole means of
feedback, outside peer assessment) was investigated by
\citet{farnqvist_competition_2016} within a Data Structures
Course.
%
This study investigated how introducing a competitive element to a
Data Structures and Algorithms \ac{CS} course might help students to
learn. The survey took place over two runs of the course over 2011 and
2012. In the 2011 course, lab contests were graded manually, and a
voluntary contest used an automated grader. In the 2012 course, the
lab assignments were marked using an automated grader for testing
correctness and efficiency, and by human lab assistants to check code
quality. The automated grader in use was
\textit{Kattis}~\cite{enstrom_five_2011}.
% 
This experiment showed the attitudes of some students towards the use
of an automated grader. Students found this automated grading to be
more fair than the manual equivalent. The study also found that
``students put in more effort in the course thanks to automated
assessment''.  The study found that competition positively affected
student behaviour. The element that was most useful for the contest
aspect of this was the autograder. The automated marking and tracking
of marks proved useful in motivating students, by offering a
leader-board.

%% Mostly from McGregor_BSc-2017
\citet{Rue+Hic+Par+Lev+Mem+Pla+Mar_CCS-2016} introduce the programming
contest \emph{Build-it, Break-it, Fix-it} that uses the concept of a
peer security testing to judge the success of various programming
assignments. This success is measured both in terms of general
correctness and specifically in the context of security. Although the
contest was used in the context of a \ac{MOOC}, its aim is
specifically targeting security and therefore could not directly be
employed for educational programming peer-assessment.

The contest requires participants to \textit{build} a working solution
that matches functional specifications, as well as being as secure as
possible. If the solution can be built and the functionality verified
by an automated system, then the team can move on to the next
stage. Following this, testers will attempt to \textit{break} the
security of the solution, and points will be allocated according to a
zero-sum-game. The builders can then gain points by \textit{fixing}
their solutions.
Throughout the contest, there is a central web-based infrastructure
that manages running the contest website (with scoreboards, etc.),
listening to builder git repositories, and also running and recording
test results.

Compare with Code Hunt and other coding-gaming platforms \cite{CHESE}.


\section{Conclusion}


\subsection{Future Works}


\begin{itemize}
\item Git-integrated as is \cite{Rue+Hic+Par+Lev+Mem+Pla+Mar_CCS-2016}
\item Improved security
\end{itemize}





%% Acknowledgments
\begin{acks}                            %% acks environment is optional
                                        %% contents suppressed with 'anonymous'
This work was partly supported by a QAA funding at Heriot-Watt University.
  %% Commands \grantsponsor{<sponsorID>}{<name>}{<url>} and
  %% \grantnum[<url>]{<sponsorID>}{<number>} should be used to
  %% acknowledge financial support and will be used by metadata
  %% extraction tools.
  % This material is based upon work supported by the
  % \grantsponsor{GS100000001}{National Science
  %   Foundation}{http://dx.doi.org/10.13039/100000001} under Grant
  % No.~\grantnum{GS100000001}{nnnnnnn} and Grant
  % No.~\grantnum{GS100000001}{mmmmmmm}.  Any opinions, findings, and
  % conclusions or recommendations expressed in this material are those
  % of the author and do not necessarily reflect the views of the
  % National Science Foundation.
\end{acks}

%% Bibliography
\bibliography{Peer-testing}


%% Appendix
%\appendix
%\section{Appendix}
%Text of appendix \ldots

\begin{acronym}[MOOC]
  \acro{CAP}{Computerised Assessment with Plagiarism}
  \acro{CS}{Computer Science}
  \acro{MOOC}{Massive Open Online Course}
  \acro{SML}{Standard ML}
  \acro{TDD}{Test-Driven Development}
  \acro{VLE}{Virtual Learning Environment}
\end{acronym}

\end{document}
