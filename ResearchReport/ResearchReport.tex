\documentclass[a4paper,11pt]{report}

\usepackage[utf8]{inputenc}
\usepackage{graphicx}
\usepackage{setspace}
\usepackage{datetime}
\usepackage{cite}
\usepackage{geometry}
\geometry{
 a4paper,
 left=1in,
 top=1in,
}
\newcommand{\titles}{\\\vspace{1cm}}

\pdfinfo{%
  /Title    (Deliverable 1: Final Year Dissertation)
  /Author   (Leon McGregor)
}

\begin{document}

{\centering\Large
\includegraphics[width=0.4\textwidth]{hwlogo}\titles
{\scshape\LARGE Heriot-Watt University\titles}
{\huge\bfseries Web platform for code peer-testing\titles}
Deliverable 1: Final Year Dissertation\titles
L\'eon \textsc{McGregor} - H00152968\titles
{\large\textit{Supervisor}\\}
%add a second reader when you get the name
Manuel \textsc{Maarek}\\
\vfill
}

\pagebreak

\tableofcontents


\pagebreak
\doublespacing


\section*{Abstract}
This document will detail the current research environment surrounding the use of peer tesing of code... This document will also set out the plan for the implementation and evaluation of a peer testing framework that students can use to test their code.

\vfill

\section*{Declaration}
I, L\'eon McGregor confirm that this work submitted for assessment is my own and is expressed in my own words. Any uses made within it of the works of other authors in any form (e.g., ideas, equations, figures, text, tables, programs) are properly acknowledged at any point of their use. A list of the references employed is included.\par
Signed: L\'eon McGregor\par
Date: \today

\pagebreak

\pagestyle{headings}
\markright{L\'eon McGregor - Deliverable 1: Final Year Dissertation}




\section{Introduction}
The aim of this project is to investigate and develop a Web Platform \& Web Application which will allow for peer testing of programming code. This system should aim to provide an easy way for teacher to request programming assignments, and for students to test each others code. While doing all this, the system should provide easy to understand feedback, to enhance the learning experience during peer testing.\\

Note the aims and objectives here.

\chapter{Background}
There has been lots of work done in looking into the use of peer testing.
\section{Peer Assessment as a learning tool}

\section{Peer Assessment and Program Security}
Much work has been done in using the concept of Peer Assessment as a tool to learn security.
\subsection{BiBiFi}
Build-it, Break-it, Fix-it is a programming contest that uses the concept of Peer Assessment to judge the success of various programming assignments, both in terms of general correctness and specifically in the context of security.
BiBiFi paper(s)\\


\subsection{In the Classroom}
A study was conducted by Hooshangi, Weiss and Cappos \cite{hooshangi_can_2015} which investigated how the use of a security-based peer testing would help with teaching how to program securely. By setting assignments which would require the students to build both 
security mindset and students

\section{...}


\chapter{Development Plan}

\section{Requirements}

\section{Design}

\section{Evaluation Strategy}

\chapter{Project Management}

\bibliography{lib}{}
\bibliographystyle{plain}

\section*{Appendices}

\end{document}
