\documentclass[a4paper,11pt]{report}

\usepackage[utf8]{inputenc}
\usepackage{graphicx}
\usepackage{setspace}
\usepackage{datetime}
\usepackage{cite}
\usepackage{geometry}
\geometry{
 a4paper,
 left=1in,
 top=1in,
}
\newcommand{\titles}{\\\vspace{1cm}}
\newcommand{\cn}{\textbf{[Citation Needed]}}

\pdfinfo{%
  /Title    (Deliverable 1: Final Year Dissertation)
  /Author   (Leon McGregor)
}

\begin{document}

{\centering\Large
\includegraphics[width=0.4\textwidth]{../hwu.png}\titles
Deliverable 1: Final Year Dissertation\titles
{\huge\bfseries Web platform for code peer-testing\titles}
L\'eon \textsc{McGregor} - H00152968\titles
{\large\textit{Supervisor}\\}
Manuel \textsc{Maarek}\titles
{\large\textit{Second Reader}\\}
Firstname \textsc{Surname}\\
\vfill
}

\pagebreak

\tableofcontents


\pagebreak
\doublespacing


\section*{Abstract}
This project aims to build a protoype for a peer assessment website. The website would be used in an educational environment, for example in a Computer Science course, and be used to let students easily perform peer assessment. This document will detail some of the existing relevant research, and detail a plan for development of the new website.

\vfill

\section*{Declaration}
I, L\'eon McGregor confirm that this work submitted for assessment is my own and is expressed in my own words. Any uses made within it of the works of other authors in any form (e.g., ideas, equations, figures, text, tables, programs) are properly acknowledged at any point of their use. A list of the references employed is included.\par
Signed: L\'eon McGregor\par
Date: \today

\pagebreak

\pagestyle{headings}
\markright{L\'eon McGregor - Deliverable 1: Final Year Dissertation}




\section{Introduction}
The aim of this project is to investigate and develop a website which will allow for peer assessment within a computer science course. This website will aim to provide an easy way for lecturers to request programming assignments, and for students to respond by submitting completed assignments, and then testing each others code. The peer assessment will allow the students to get fast and easy to understand feedback for assignments, and could be used as a starting point for grading of assignments.\par
The website will need to handle this code in a secure manner, to prevent either intentional or accidental malicious code from running.\par
Much research has been already completed in the study of Peer Assessment, from the front of being used as an educational tool, to being used as a basis for competition between programmers. Some relevant research into this will be analysed here.

%summarize the research here
%Note the aims and objectives here.

\chapter{Background}
There has been lots of work done in looking into the use of peer testing. Introduce the various things I looked at here, as an overview.
\section{Peer Assessment}
\subsection{Anonymous Peer Assessment}
When developing a peer assessment system, or an assessment system of any kind, it is important to know whether or not to reveal the identities of those being assessed. Very often, assessment marking is done anonymously\cn.A study was conducted\cite{li_role_2016} that aimed to investigate just how effective anonymity is when it comes to making peer assessment more effective, and whether any negative impact from a lack of anonymity can be mitigated.\par
This quasi-experimental study was conducted with some in-training teachers, and aimed to see which is the most effective method of / using / while conducting a peer assessment exercise: Having assessors and assessees know each others identities, remain anonymous, or know identities while having received training.\par
One concern I have with this study is that it didn't cover the case of being anonymous and getting training. This was because the training was intended as a fallback for when anonymity is not possible. But this does invite the question of just how effective would anonymity be if training were offered as well.\par

\subsection{}

\subsection{In the Classroom}
A study was conducted by Hooshangi, Weiss and Cappos \cite{hooshangi_can_2015} which investigated how the use of a security-based peer testing would help with teaching how to program securely. By setting assignments which would require the students to build both 
security mindset and student

\section{Implementations}

\subsection{Automated Testing}
One interesting feature that could be implemented in a peer assessment website would be to integrate an automated testing framework. This would be able to give immediate (though limited in detail) pass/fail information, without having to wait for peers to add their own detailed feedback. The effectiveness of an automated testing tool (in this case not integrate into a competitive environment, not a peer assessment one) was investigated \cite{farnqvist_competition_2016}.\par
This study investigated how introducing a competitive element to a Data Structures and Algorithms CS course might help students to learn. The survey took place over two runs of the course over 2011 and 2012. In the 2011 course, lab contests were graded manually, and a voluntary contest used an automated grader. In the 2012 course, the lab assignments were marked using an automated grader for testing correctness and efficiency, and by human lab assistants to check code quality.\par
This test showed the attitudes of some students towards the use of an automated grader.

\subsection{BiBiFi}
Build-it, Break-it, Fix-it\cite{ruef_build_2016}\cite{ruef_build_2015} is a programming contest that uses the concept of a Peer testing system to judge the success of various programming assignments, both in terms of general correctness and specifically in the context of security. This is somewhat different to the aim of an educational peer assessment website, however it is worth mentioning as ... why?
BiBiFi paper(s)\par

\subsection{apparmor}

\section{...}


\chapter{Evaluation Strategy}
The aim of this dissertation is to .. .uhh,,,. hmm

\chapter{Development Plan}

\section{Requirements}

\section{Design}


\chapter{Project Management}

\bibliography{../Dissertation}{}
\bibliographystyle{plain}

\section*{Acknowledgements}

\section*{Appendices}

\end{document}
